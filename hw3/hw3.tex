\documentclass{article}
\usepackage{amsmath, amssymb, graphicx}

\title{HW3: Fourier Transform Proofs}
\author{Yumin Chiang}
\date{\today}

\begin{document}

\maketitle

\section{Proof of Inverse Fourier Transform}
We want to prove the following inverse Fourier transform formula:
\begin{equation}
    \psi(x) = \int_{\mathbb{R}^K} \Psi(f) e^{j 2\pi f^T x} df.
\end{equation}

From the lecture notes (page 10), we use the result:
\begin{equation}
    \int_{-\infty}^{\infty} e^{j 2\pi f_0 x} dx = \delta(f_0),
\end{equation}
where $\delta(f_0)$ is the Dirac delta function.

Taking the Fourier transform definition:
\begin{equation}
    \Psi(f) = \int_{\mathbb{R}^K} \psi(x) e^{-j 2\pi f^T x} dx,
\end{equation}
we substitute this into our equation:
\begin{align*}
    \int_{\mathbb{R}^K} \Psi(f) e^{j 2\pi f^T x} df &= \int_{\mathbb{R}^K} \left( \int_{\mathbb{R}^K} \psi(y) e^{-j 2\pi f^T y} dy \right) e^{j 2\pi f^T x} df.
\end{align*}
Interchanging the order of integration:
\begin{equation}
    \int_{\mathbb{R}^K} \psi(y) \left( \int_{\mathbb{R}^K} e^{-j 2\pi f^T y} e^{j 2\pi f^T x} df \right) dy.
\end{equation}
Using the result $\int e^{j 2\pi f (x-y)} df = \delta(x-y)$, we obtain:
\begin{equation}
    \psi(x) = \int_{\mathbb{R}^K} \psi(y) \delta(x-y) dy = \psi(x),
\end{equation}
which completes the proof.

\section{Proof of Convolution Theorem}
We aim to prove:
\begin{equation}
    \phi(x) = \psi(x) \ast h(x) \leftrightarrow \Phi(f) = \Psi(f) H(f).
\end{equation}

By definition, the convolution is:
\begin{equation}
    \phi(x) = (\psi \ast h)(x) = \int_{\mathbb{R}^K} \psi(y) h(x-y) dy.
\end{equation}

Taking the Fourier transform:
\begin{align*}
    \Phi(f) &= \int_{\mathbb{R}^K} \phi(x) e^{-j 2\pi f^T x} dx \\
    &= \int_{\mathbb{R}^K} \left( \int_{\mathbb{R}^K} \psi(y) h(x-y) dy \right) e^{-j 2\pi f^T x} dx.
\end{align*}
Interchanging the order of integration:
\begin{equation}
    \int_{\mathbb{R}^K} \psi(y) \left( \int_{\mathbb{R}^K} h(x-y) e^{-j 2\pi f^T x} dx \right) dy.
\end{equation}
Using the substitution $u = x - y$, we get:
\begin{equation}
    \int_{\mathbb{R}^K} \psi(y) e^{-j 2\pi f^T y} \left( \int_{\mathbb{R}^K} h(u) e^{-j 2\pi f^T u} du \right) dy.
\end{equation}
Recognizing the Fourier transforms, we obtain:
\begin{equation}
    \Phi(f) = \Psi(f) H(f),
\end{equation}
which completes the proof.

\section{Conclusion}
These proofs establish the fundamental properties of Fourier analysis, specifically the inverse transform and the convolution theorem. The mind-maps should illustrate these derivations clearly.

\end{document}
